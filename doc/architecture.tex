%!TEX root = paper.tex

% Software architecture, not CNN-architecture
\section{Architecture}\label{sec:arch}

This section describes the general architecture of our software solution and reasoning for choosing specific tools.
We begin by describing the website front end which is visible to the user.
From there on we move to the server's back end and finally into our business logic consisting mainly of independent python programs.

We chose python as our main programming language, because of its extensive usage and support in machine learning -- especially the TensorFlow framework.
We decided to build a website as our main user interface and subsequently chose the Django Framework to easily integrate our python programs into the server back end.
Considering the attack algorithms, we selected the CleverHans library because it provides a unified interface to a variety of attacks.
By using a website as the front end, our user interface is largely platform independent and nicely separated from our back end, which we provide via a REST API.
Also, the web context allows us to use existing techniques to provide a visually appealing and responsive design.

Currently, our website is a local single user system, which we distribute as a docker container.	
We leave the thorough security evaluation and multi-user management to future extensions and focus more on generating strong adversarial examples.
Though, using docker containers and having a clear separation between the front end, the HTTP request management as well as our business logic, our architecture makes these extensions feasible.
This separation has been strictly implemented which furthermore allows experienced users to connect to the docker container's TTY and use the python script back end directly without starting the web server.

\subsection{Front End}\label{subsec:frontend}

The front end is divided into two categories: \enquote{Models} (model creation) and \enquote{Attack}.
In the former category, the user is able to train, steal, upload and generally manage models. 
The latter is then used to configure and start attacks on a model.
All information in this subsection relates to the client-side part of our application: HTML documents, CSS style sheets, JavaScript files and other static content.

The structure of the front end can be seen in Figure~\ref{fig:rest_api}.
This also contains the available GET and POST requests provided in the back end REST interface which are triggered by the front end upon specified actions like a button click or a timeout.
From a conceptual point of view, the front end can be seen as a pipeline: The user is guided from creating a substitute model to generating adversarial examples.
This manifests in navigational buttons providing shortcuts and automatic redirections to the correct page on certain actions.
For example, the user is automatically redirected to the related details page of a process after starting it (on a different page).

\begin{figure}
	\begin{forest}
	for tree={
		font=\ttfamily,
		grow'=0,
		child anchor=west,
		parent anchor=south,
		anchor=west,
		calign=first,
		edge path={
			\noexpand\path [draw, \forestoption{edge}]
			(!u.south west) +(7.5pt,0) |- node[inner sep=1.25pt] {} (.child anchor)\forestoption{edge label};
		},
		before typesetting nodes={
			if n=1
			{insert before={[,phantom]}}
			{}
		},
		fit=band,
		before computing xy={l=15pt},
	}
	[/
		[welcome.html
			[/save\_api\_key (POST)]
		]
		[models
			[/overview.html]
			[/training.html]
			[/details.html]
			[/model\_info (GET)]
			[/proc\_info (GET)]
			[/deletemodel (POST)]
			[/uploadmodel(POST)]
			[/start\_training (POST)]
			[/abort\_training (POST)]
		]
		[attack
			[/overview.html]
			[/attack.html]
			[/details.html]
			[/proc\_info (GET)]
			[/list\_images (GET)]
			[/classify (GET)]
			[/start\_attack (POST)]
			[/delete\_proc (POST)]
		]
	]
\end{forest}
	\caption{A hierarchical overview of the provided REST interface.}\label{fig:rest_api}
\end{figure}

\subsubsection{Design Principles}
The main goal for the front end is a visually appealing and responsive design.
This includes the scalability of the website across different screen sizes and even device types.
Therefore, we utilize the flexbox model and responsive CSS classes to dynamically organize DOM elements on the screen.
As an example, the navigation bar changes its appearance when the view port becomes too small, condensing the navigation bar items in a drop down menu.
We also leverage the twitter bootstrap framework for a scalable base design\footnote{\url{https://getbootstrap.com/}} as well as a bootstrap wrapper theme\footnote{\url{https://bootswatch.com/darkly/}} which we found suitable for a hacking-related competition.

A second requirement is a user-friendly flow.
This includes a unified layout across multiple subdomains and expressive icons for which we used the free icons from fontawesome\footnote{\url{https://fontawesome.com/}}.
Moreover, this design goal implies low latencies after clicking a button and an immediate visual feedback, which is solved by our REST API and asynchronous requesting(see~\ref{subsec:webserver}).
When a time-consuming action is issued, the back end server starts an asynchronous process and allows for polling its status.
This enables the front end to provide immediate feedback of the internal status.
The user's orientation on each page is further supported by hovering effects in potentially large tables (amongst other things).
Finally, a user-friendly design also means preventing the user from taking possibly inadvertent actions like deleting a trained model which took multiple hours to train.
We thoroughly implemented these design goals in all parts of the website.

As previously mentioned, the subdomains share a base layout and some other properties like included style sheets and scripts.
To avoid code duplicates, the HTML documents are dynamically generated using the Django template system and the provided template inheritance feature.
This allows for having a root template which others can inherit from.
As a consequence, a change in e.g. the navigation bar would only affect a single file, which reduces the risk of code inconsistencies.
The templates themselves are filled according to a given context: For example, the table of available models is dynamically created with respect to the prevalent models in the docker container.
To implement this, django offers a simple yet expressive domain specific language for defining variables, loops and conditions.
Thereby, these templates also help at shortening HTML documents.

\subsubsection{Welcome Page}
The welcome page is intended to show an inviting and motivating starting page with a quick navigation to the two main parts of the front end.
It explains in simple words what to expect there and employs a minimal design.
On visiting this page, a background check is performed: The API key necessary for the remote predictions is validated by making a test request.
If no API key has been provided yet, the user is asked to type it.
Once a valid key has been submitted, the dialog disappears and the user free to navigate subsequent pages.
This measure is an important step because many of the features of our program require a valid API key.

\subsubsection{Models -- Overview}
As discussed in Section \ref{sec:methodology}, for an attack to be possible, we need to train a local substitute model.
The model overview tab contains all available models as well as their meta-data (name, size, date of last modification, architecture).
Clicking on the info icon shows a modal containing a detailed overview about the model's layers.
The user can choose to attack one of the trained models and is redirected to the attacking section accordingly. 
Furthermore, the user is able to upload own models on this page.
The trash icon triggers deleting a model.
Referring to the above described design principles, the website prompts for a second approval before actually performing the request to prevent deleting a model by accident.

\subsubsection{Models -- Training}
This page enables the user to configure and start a new training phase of a substitute model. 
Since training a model is very resource intensive, we prevent multiple simultaneous training processes.
The page itself consists of a form which collects all training parameters.
We chose to provide default values which yielded the best results in our experiments, effectively supporting a potentially less-experienced user.

\subsubsection{Models -- Details}
This page displays the details of the current training process and enables a user to abort the running training phase.
To show the live console output, a script is polling the internal state of the respective standard output (stdout) file descriptor via our REST API. 
If there is no process in progress, an appropriate message is shown.

\subsubsection{Attack -- Overview}
Similar to the model overview, this page contains a table to manage the different attacks.
The shown entities in this context are the performed runs: Each started attack is identifiable by its process ID (PID) and the console output can be reviewed live or in retrospective.
By clicking on a PID, the respective console output is shown.
If an attack run is not needed anymore, it can be removed from the table using the trash icon.

\subsubsection{Attack -- Attack}
This tab offers starting a new attack on a source image.
The form fields update according to the selected attack algorithm because not all approaches use the same set of parameters.
Also, the user is prompted to upload an image which is then taken to attack the selected model.
Starting multiple attacks simultaneously is supported here since these are less resource intensive.
An error alert may show up if any invalid form data had been passed.
For example, loading a source image which is not 64 by 64 pixels in size results in such an error.

\subsubsection{Attack -- Details}
This tab works almost exactly like the model details except for that it additionally shows the classification result of the source and generated images.
Thus, it enables the user to evaluate her attack after it has finished.
Also, the generated images are shown on this page which enables the user to download them by right-clicking and saving them.

\subsection{Django Server}\label{subsec:webserver}
The Django server acts as a blend of web and application server.
It serves static content like style sheets and HTML documents and responds to HTTP requests.
We leveraged Django's middleware capabilities to automatically serve the static content and dispatch HTTP requests.
As a result, we only had to implement the responses to the HTTP GET and POST requests.
In general, we decided to implement a REST API in which GET requests are used to ask for the state of certain resources which are identified by an URI.
These requests are idempotent, meaning that successive identical GET requests yield the same server responses (as demanded by RFC 7231\cite{rfc7231}).
All state-changing actions are implemented as POST requests, for which an CSRF protection token is sent along with the generated HTML page.

The request handling code does not contain any further logic except for parsing the requests and some basic input validation which is used to display the respective error message in case of invalid request data.
We decided for a design with server-side input validation which would be a critical security requirement in future extensions supporting hosting the website over a network.
These handlers issue other python scripts, which also work in a standalone mode.
This is the manifestation of the previously described strict separation of duties.

Regarding the code separation, each main component is realized as a separate \enquote{Django App}: This is Django's mechanism for grouping sub-components.
We identified three main components in our software which are implemented as separate Django Apps: welcome, models and attack.
This grouping is reflected in the HTML templates, JS scripts and CSS files, which increases the interchangeability of components and the enforcement to use clear interfaces.

For actions that require a lot of processing like training or attacking a model, we employed a special mechanism for handling asynchronous sub-processes.
In order to organize these, we leverage the existing operating system process management.
An overview about the information flow is presented in Appendix~\ref{app:doublefork}.
When the user issues a start command (either training or attack), a new process is spawned and the website receives an immediate response whether starting the process has been successful.
Then, a JavaScript \enquote{setInterval} function periodically polls the state of the process.
To test whether a process is still running, the Django middleware emits a KILL(0) system call which is successful if there exists a running process with the specified PID and errors otherwise.
However, we need to apply a small trick to be able to use this mechanism which is known as \enquote{double fork}: To spawn an asynchronous worker process, we first fork an intermediate process that again forks the target process, returns its PID to the parent (Django handler) and then exits.
Due to the behavior of the operating system, the worker process is now a child of the init process (PID 0), which takes care of terminating the process after completion automatically.
This is important to prevent zombie processes because these would remain as \enquote{defunct} processes in the process table of the operating system, until the parent process (Django) exits.
Since we don't want to restart the web server because a sub-process has finished, we applied this double fork technique.
The advantage of this setup is that we leverage existing operating system mechanisms and don't have to re-implement this ourselves which might not cover all edge cases.

\subsection{Python Back End}\label{subsec:backend}
The python back end represents our business logic and is realized as a set of standalone scripts for the main tasks, some auxiliary scripts and a small configuration file which defines common variables like the path to the cache directory.
Considering the standalone scripts, each of them realizes a high level function like starting a training, an attack or crawling the remote predictions for a dataset.
They expose all internal parameters as command line arguments such that they can be either called by hand or by the above described sub-processing mechanism.
Whenever there are commonly used functions between these scripts, an auxiliary script has been created to avoid code duplication which implements the DRY\footnote{DRY = \enquote{don't repeat yourself}} software developing principle.

An example for an auxiliary script is \enquote{gtsrb.py}, which collects all properties of the reference dataset and provides functions to retrieve the data as well as plotting distributions over the different classes.
This encapsulates all information and functions belonging together into a single script and provides them over an interface.
If we were to manipulate data on a different reference dataset in the future, we would only have substitute this file with a similar one containing the information for the new dataset.
As a consequence, many of the internal scripts are interchangeable which greatly improves the maintainability and extensibility of our program.

Regarding the choice of programming language, there have been three main factors which emphasized why python would be the best choice in our case:
\begin{description}
	\item[Scripting Language] This type of language allows for writing very expressive and succinct code, which helps at implementing features fast while avoiding boiler plate code. 
	A shorter source code is easier to maintain and fewer lines of code are also likely to contain fewer bugs~\cite{jain2017clairvoyant}. 
	Considering the small time frame of the competition, this outweighs possible drawbacks like the missing error checking at compile time.
	\item[Packet Management System] The python package management system \enquote{pip} incorporates all required dependencies in the default sources and allows for easily maintaining them on a project level using virtual environments. 
	Utilizing widely used  and mature packets like Django or CleverHans, we are confident that these dependencies impose only a small risk of introducing bugs.
	\item[Developer Experience] Developers write better code in languages they are highly familiar with.
	Since this is the case for both team members, we preferred this choice over other equally good solutions.
\end{description}

%TODO reference to \ref{subsec:cwl2_mod}
%TODO mention other scripts
%TODO write about augmentation

\subsection{Docker Setup}\label{subsec:docker}
Docker is a commonly used containerization and virtualization engine.
It provides the opportunity to bundle a program and all its dependencies into a single container which then may be run on different operating systems.
The result is an easy to install software with the only requirement using the same kernel as the host.
Also, completely uninstalling a program with all its dependencies becomes an easy task with docker containers.
This technology therefore greatly simplifies distributing software and helps avoiding situations in which it works on a developer's machine but not on the user's.
For the final submission, we choose to offer two options.
First, we provide a \enquote{Dockerfile} and the git repository, which the user may use to create a docker image herself and instantiate a container.
Second, we also distribute a ready-to-run docker container incorporating a copy of the repository.

Having a closer look on the container, one might recognize that it is distributed without the reference dataset (GTSRB) and external dependencies (like pip packages and twitter bootstrap).
We decided to reduce the size of the container and separate the code from the data as much as possible.
Subsequently, there exists a setup shell script which will download all required data and put it in a so called \enquote{DATA\_ROOT} on startup of the container.
This data root is a directory that can be mounted into the docker container such that, on a potential software update, the user would only need to replace the docker container and can then immediately continue working as before.
A further benefit of this is that all relevant information concerning the program's state is also saved there which implies that the user could update the software without data loss.
Considering that training models and fetching predictions is a very time-consuming task, we wanted to ensure that the user has to perform these tasks not more than once.

Finally, the virtualization aspect of docker enables us to leverage operating system features like system calls and allowing a web application to write files without the risk of harming the host environment.
For example, docker employs strict policies preventing containers from accidentally (or intentionally) modifying host files with the exception of mounted directories (like the data root).
From a security perspective, it is also valuable to only expose a single port to the host network which is the one used to serve the website.
