%!TEX root = paper.tex

\section{Discussion}\label{sec:discussion}
In the following, the overall quality of our solution to the InformatiCup challenge is discussed.
Without recapitulating all previously mentioned information, we point out three highlights of our work as well as aspects which leave room for further improvement.

In general, we find our solution to be sound across the whole concept from the theoretical approach up to the implementation in software.
State-of-the-art insights from current literature have been combined and systematically adjusted to our use case.
We therefore consider this work to be of scientific value, acting as a reference case study for a real-world setting.

A second highlight is in our opinion the user-friendly and visually appealing front end.
Though this has not been a formal requirement, the challenge explicitly states that creative extensions of requirements are welcome.
We considered it as most valuable to open up the potential user base of our program from command line experts to everyone who is able to install a docker container and open a web browser.
This significantly increases the potential relevance of the program while improving the overall user experience.

Third, we would like to emphasize the value of the custom extensions to training and attack algorithms in our solution.
They build on scientifically founded ideas and extend them to better fit the given context.
For example, by implementing the modified CWL2 attack, we extended the scope of the original challenge to a real-world scenario in which our perturbations could actually be applied.
Also, the robust CWL2 attack modification allows for graffiti-like perturbations which cover the use case of limiting the adversarial modifications to a certain shape, which again is useful for printing them as stickers.

Due to the limited time frame of the competition, there also some parts of the project which can be further improved in the future.
As an example, we were both familiar with each other's coding style and implicitly followed the same conventions without formalizing them.
Though the resulting software is still consistent, there are only source code comments where we considered them to be sensible.
We tried to write very expressive code but are aware that a third developer later working on our code base might disagree on this as it is a highly subjective perception.

Considering the front end, we optimized the user experience along the natural flow of entering the API key, training a model and attacking it.
However, it is for example currently possible to skip the API key dialog by triggering the respective JavaScript function by hand in the browser's developer tools or by manually entering a different subdomain.
Additionally, we did not consider these types of actions from a security perspective because the user would only harm herself in our single-user application.
We prevent all actions a user might unintentionally trigger like deleting models by clicking on a wrong button but we generally assume a benevolent user.
This might become a drawback when extending this application to the multi-user case.
Though our architecture naturally supports this, the actual implementation would impose some additional effort.
